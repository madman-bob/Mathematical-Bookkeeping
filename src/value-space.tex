\section{Value}

\subsection{Money Space}

To start talk of bookkeeping, we fist need a thing to keep track of -- some concept of money.
What ``money'' actually means to us depends on our specific circumstances, but let's call the set of all possible monies $\M$.

\begin{example}
    For example, if the only money we accept is the GBP, then our money space would be

    \begin{equation}
        \label{m-def}
        \M = \{ \pounds 0.00, \pounds 0.01, \pounds 0.02, ... \}.
    \end{equation}
\end{example}

The smallest amount of money we can have is nothing,
and the primitive operation we can do with our money is to add it together.
If we have one pile of money, and another pile of money, then we can put them together into a bigger pile of money.

So we want some operation $+$, that allows us to add values together in the way that we expect,
and some value $0$, which represents the value of nothing.
Mathematically,

\begin{definition}
    A \emph{value space} $\M$ is a commutative monoid.
\end{definition}

I'm glossing over the precise definition of a commutative monoid here, as, from an accounting perspective,
everything works as you expect from the analogy of piles of money on the floor.
More details on monoids may be found in any introductory work on algebraic structures, such as \cite{monoids}.

% Do we want to include 0?
%   Doesn't matter. Simplifies some things to include it, but can work around it
%     Get 0 for free in value space, as 0 = (a, a), for some `a`.
%     Normaly identify `a` with (a, 0), but could instead use (a + a, a)
% Do we want two values to collapse to the same value in the Pacioli group?
%   eg. Define 0' by 0' + a = a = a + 0' \forall a != 0, and 0' + 0 = 0' = 0 + 0'
%   Then 0' != 0 in \M, but 0' == 0 in \V.

\subsection{Value Space}

But companies do more than just take money\footnotemark.
\footnotetext{We hope}
They also need to purchase raw materials, and pay employees.
So we need some way to represent not only the money, but also whether it's a gain, or a loss.

Naively, we want to add a $-$ operation to $\M$, so that we can take some money from another amount.
But in order for this to make sense, we need to add negative numbers to our space as well (so we can do things like $\pounds 0 - \pounds 10$).

Historically, western mathematicians were uncomfortable with the concept of negative numbers \cite{arithmetica},
and so instead the concepts of debits and credits were used.
These two methods are, in fact equivalent, as shown by the Pacioli group, as defined in Ellerman, \emph{On Double Entry Bookkeeping} \cite{ellerman-tmt}.

\begin{definition}
    The \emph{Pacioli group}, $P$, of a commutative monoid $M$ is the set of ordered pairs $\taccount{a}{b}$ with both elements in $M$,
    under the equivalence relation $\taccount{a}{b} R \taccount{c}{d}$ when $a + d = b + c$.

    Further, we define $+$ pointwise on $P$, define unary $-$ by $- \taccount{a}{b} = \taccount{b}{a}$, and binary $-$ by $u - v = u + (- v)$,
    for all $a, b \in M$, and $u, v \in P$.
\end{definition}

\begin{definition}
    A \emph{value space} $\V$, is the Paciloi group of a money space $\M$.
\end{definition}

In a pair $\taccount{a}{b}$ in a value space $\V$, we think of $a$ as a debit, and $b$ a credit.

\begin{notation}
    We identify a money $a$ with the pair $\taccount{a}{0}$, so that we can write
    \begin{equation*}
        \taccount{a}{b} = a - b.
    \end{equation*}

    This is equivalent to identifying the Pacioli group with its debit isomorphism, as defined in Ellerman, \emph{Economics, Accounting, and Property Theory} \cite{ellerman-eapt}.
\end{notation}

Note that the choice of identifying a money $a$ with the debit component of the pair is completely arbitrary,
and we could instead use the credit component $a \mapsto \taccount{0}{a}$.
All that would do would flip all the $+$ and $-$ signs.

\begin{example}
    Using \eqref{m-def} for $\M$, our value space is
    \begin{equation*}
        \V = \{ ..., -\pounds 0.02, -\pounds 0.01, \pounds 0.00, \pounds 0.01, \pounds 0.02, ... \}
    \end{equation*}
    where a positive amount represents a debit, and a negative amount a credit.
\end{example}

\begin{example}
    If you have a $\pounds 10$ note, and owe a friend $\pounds 5$, then your net worth is
    \begin{equation*}
        \pounds 10 - \pounds 5 = \pounds 5
    \end{equation*}
    leaving you with $\pounds 5$.

    If instead you owed your friend $\pounds 20$, then your net worth is
    \begin{equation*}
        \pounds 10 - \pounds 20 = -\pounds 10
    \end{equation*}
    leaving you $\pounds 10$ in debt.
\end{example}
