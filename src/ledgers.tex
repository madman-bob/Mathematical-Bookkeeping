\section{Ledgers}

Now let's put it all together.
A financial system consists of multiple transactions, so we need some sort of collection to hold them all,
which we'll call our ledger.

We could use a list to represent our ledger, but we've already captured our concept of order in temporal transactions.
As we don't need to consider order again, we could use a set,
but we could, in theory, have two of the exact same transaction at the exact same time.
So we want a set with appropriate counting of duplicates.

\begin{definition}
    A \emph{ledger} is a finite\footnotemark multi-set of temporal transactions.
\end{definition}
\footnotetext{We could probably be less restrictive than this, but all real-world examples are finite}

This ledger represents all the transactions in a financial system.
We can now speak of things like the balance of an account, and interaction of accounts.

\begin{definition}
    The \emph{balance} $b$ of an account $A$, in a ledger $L$, is a function $\Ti \to \V$,
    where the balance at time $\tau$ is given by
    \begin{equation*}
        b(\tau) = \sum_{t \in L} t(\tau)_A.
    \end{equation*}
\end{definition}

\begin{definition}
    We say two accounts $A$ and $B$ have \emph{interacted} in a ledger $L$,
    if there is a temporal transaction $t \in L$, which involves both $A$ and $B$.
\end{definition}
