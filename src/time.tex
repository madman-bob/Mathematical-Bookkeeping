\section{Time}

So we now have transactions, and a way of combining them (addition).
But the order in which these transactions occur is still important, as (for example) you can't sell goods before you own them.
We could just take a list of the transactions, but then we forget when exactly something happened,
which may be important for reporting purposes.

So we probably want to record our transactions with some sort of timestamp.
So we need some concept of time.

\begin{definition}
    Let our set of \emph{times} $\Ti$ be a total order.
\end{definition}

But if we want to add together multiple transactions with different timestamps,
then we need to extend our space of transactions to allow transactions that take different values at different times.

\begin{definition}
    Let a \emph{temporal transaction} be a function $\Ti \to \Tr$ from time to transactions.

    Define the various transaction operations on temporal transactions pointwise.
    We say a temporal transaction $t$ \emph{involves} an account $A$, if there is a time $\tau$, with $t(\tau)_A \neq 0$.
\end{definition}

The most natural class of temporal transactions are changes that takes place instantaneously, at a certain point in time.
Let's call these step transactions.

\begin{definition}
    For a transaction $t$, and a time $\tau$, define the \emph{step transaction} $t^\tau$ to be the temporal transaction given by

    \begin{equation*}
        t^\tau(\tau') =
        \begin{cases}
            0 & \tau' < \tau \\
            t & \tau' \ge \tau
        \end{cases}
    \end{equation*}

    for all times $\tau'$.
\end{definition}

Note that classic double-entry bookkeeping uses only step transactions, as transactions are applied instantaneously.
More complicated transactions are then represented by multiple step transactions.
Temporal transactions, on the other hand, allow a much richer description of how transactions take place.
We will consider some more complicated examples of temporal transactions later.
