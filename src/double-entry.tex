\section{What's ``double'' about double-entry?}

The name of double-entry bookkeeping comes from the recognition that transactions have both a source, and a destination.
Therefore, a standard transaction, with one source, and one destination, must be written down twice, once for each account.
In a very literal sense, the entry is doubled.

Similarly, we could recognize that, in a transaction, the value has to come from somewhere, and it has to go somewhere.
A kind of ``Conservation of Value'', if you will (to take inspiration from physics naming conventions),
or ``Fundamental Theorem of Accounting'' (to take inspiration from mathematics naming conventions).\footnotemark
\footnotetext{
Though with the Fundamental Theorems of Algebra and Arithmetic, I think mathematics has enough FTAs.
}

\begin{theorem}
    The total value in a financial system is the same before and after a transaction,
    where the total value of a system, with a financial state $f$ and set of accounts $\A$, is given by
    \begin{equation*}
        \sum_{A \in \A} f_A.
    \end{equation*}
\end{theorem}

\begin{proof}
    Let $f$ be a financial state, $t$ a transaction, and $\A$ the set of accounts.

    Then the total value of the financial system after the transaction is $\sum_{A \in \A} (f + t)_A$.
    This gives
    \begin{align*}
        \sum_{A \in \A} (f + t)_A &= \sum_{A \in \A} f_A + \sum_{A \in \A} t_A, && \text{rearranging the sum,} \\
        &= \sum_{A \in \A} f_A + 0, && \text{by definition of a transaction,} \\
        &= \sum_{A \in \A} f_A.
    \end{align*}
\end{proof}

Note that we could equally define transactions as value preserving changes in financial states,
and then derive the current definition from that.

For more complicated transactions, we can have more than two non-zero entries,
so perhaps the name ``double'' is a bit of a misnomer.
On the other hand, we still require a balance between the ``debits'' and ``credits'', which is a sort of doubling up of values.
