\section{Accounts}

\subsection{Individual bookkeeping}

If there's only one entity in all of existence, then transactions are very boring, as all that can happen is that entity giving things to themselves.
As such, we need to consider more than one entity, but how many?

We generally do bookkeeping to keep track of our transactions with the outside world.
So we could just consider two entities, $\me$ and $\nme$.

This allows us to record what transactions happen, but usually we also want to record more details than that.
We care for what purpose we're making each transaction, and we care who we're making the transaction with.
This can be represented by splitting $\me$ and $\nme$, respectively, into smaller parts.
How small those parts are depends on how much detail you want to record.

Thus, we have a finite collection of entities involved in our accounting.
Call each of these entities an ``account'', and let $\A$ be the set of all accounts.
From a mathematical perspective, accounts are just labels, and what they represent doesn't matter.
From an accounting perspective, accounts allow you to keep track of where your money is coming from, and going to.

\subsection{Distributed bookkeeping}

As a side note, nothing is requiring us to view our accounting from the point of view of any particular entity.
A collection of individuals could theoretically come to a common agreement on what an account is, and maintain their books together.
In practice, this would probably require some sort of distributed bookkeeping software
(eg. in the style of Bitcoin).
