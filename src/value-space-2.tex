\section{Multi-dimensional value space}

\subsection{Product money space}

Unfortunately, there is more than one type of money, and we can also trade in goods not generally considered as money.
These different commodities are not directly compatible, in an ``apples and oranges'' manner.

\begin{example}
    If we consider the money spaces
    \begin{equation*}
        \pounds = \{ \pounds 0.00, \pounds 0.01, \pounds 0.02, ... \}, \text{and}
    \end{equation*}
    \begin{equation*}
        \$ = \{ \$ 0.00, \$ 0.01, \$ 0.02, ... \},
    \end{equation*}
    then $\pounds 10$ is not interchangeable with any amounts of $\$$.\footnotemark
    \footnotetext{
    If your first reaction involves the phrase ``exchange rate'',
    then note that you're exchanging the $\pounds 10$ \emph{with someone}.
    So what you're actually doing is a transaction where you're purchasing some amount of $\$$ for $\pounds 10$.
    You can't change $\pounds 10$ to some amount of $\$$ without getting someone else involved.
    }
\end{example}

If we have two different money spaces, $\M_1$, and $\M_2$,
then we might want to define the money space consisting of \emph{both} the monies $\M_1$ and $\M_2$.
But to make this new space a money space, we'd also have to include all possible sums of values from both these spaces.
Mathematically, this is known as the product space (product as in multiplication, rather than something you sell).

\begin{definition}
    Given two money spaces $\M_1$, and $\M_2$, we define the \emph{product money space} $\M_1 \times \M_2$ to be the set of pairs $(m_1, m_2)$,
    with $m_1 \in \M_1$, and $m_2 \in \M_2$.

    Further, we make the product money space into a money space by defining addition component-wise.
    \begin{equation*}
        (m_1, m_2) + (n_1, n_2) = (m_1 + n_1, m_2 + n_2).
    \end{equation*}
\end{definition}

\begin{example}
    For the above money spaces of $\pounds$ and $\$$, the product money space is
    \begin{equation*}
        \begin{array}{ll}
            \pounds \times \$ = \{ \\
            & \begin{matrix}
                  (\pounds 0.00, \$ 0.00), & (\pounds 0.00, \$ 0.01), & (\pounds 0.00, \$ 0.02), & \dots \\
                  (\pounds 0.01, \$ 0.00), & (\pounds 0.01, \$ 0.01), & (\pounds 0.01, \$ 0.02), & \dots \\
                  (\pounds 0.02, \$ 0.00), & (\pounds 0.02, \$ 0.01), & (\pounds 0.02, \$ 0.02), & \dots \\
                  \vdots & \vdots & \vdots & \ddots
            \end{matrix} \\
            \}.
        \end{array}
    \end{equation*}
\end{example}

By repeating this process, we can make money spaces out of arbitrarily many smaller money spaces.

Similarly to transactions, product money spaces require you to know the shape of the whole space to be able to write down an element of it.
So we introduce analogous notation to allow you to express monies in product money spaces locally, without needing to know all components.

\begin{notation}
    We identify the monies of the two spaces with the appropriate subspaces of the product money space
    \begin{equation*}
        m_1 \mapsto (m_1, 0),
        m_2 \mapsto (0, m_2)
    \end{equation*}
    for all $m_1 \in \M_1$, and $m_2 \in \M_2$.
\end{notation}

\begin{example}
    Then, in the space $\pounds \times \$$, we can write things like $\pounds 10 + \$ 10$ to mean $(\pounds 10, \$ 10)$ as
    \begin{align*}
        \pounds 10 + \$ 10 &= (\pounds 10, \$ 0) + (\pounds 0, \$ 10) \\
        &= (\pounds 10, \$ 10)
    \end{align*}
\end{example}

\subsection{Product value space}

We can similarly define the combination of two value spaces,
and show that the value space of a product space is essentially the same as the product space of a value space.

\begin{definition}
    Given two value spaces $\V_1$, and $\V_2$, we define the \emph{product value space} $\V_1 \times \V_2$ to be the set of pairs $(v_1, v_2)$,
    with $v_1 \in \V_1$, and $v_2 \in \V_2$.

    Further, we define value space operations on $\V_1 \times \V_2$ componentwise.
\end{definition}

\begin{theorem}
    If we have two money spaces $\M_1$, and $\M_2$, with corresponding value spaces $\V_1$, and $\V_2$,
    then the value space $\V$ of the product money space $\M_1 \times \M_2$
    is isomorphic to $\V_1 \times \V_2$.
\end{theorem}

\begin{proof}
    The isomorphism is given by
    \begin{equation*}
        \taccount{(m_1, m_2)}{(m'_1, m'_2)} \mapsto \left(\taccount{m_1}{m'_1}, \taccount{m_2}{m'_2}\right).
    \end{equation*}
    Proof that this is an isomorphism is omitted, as it is long winded, but unenlightening.
\end{proof}

This shows that we can take products of spaces and value spaces of money spaces in any order, without risking confusion.

\subsection{Non-standard money types}

As well as freely adding money types, we can consider types that we do not normally think of as monies.

In classical double entry bookkeeping, the purchase of an asset decreases the balance of our cash account, and increases the balance of our assets account by the same amount.
But this really represents a transfer of money to some external entity, in exchange for that asset.
So we can instead encode the asset as a new money type, and have the transaction as a literal exchange of cash for asset.

\begin{example}
    For example, a transaction where I buy a sheep from Fred for $\pounds 100$ could be represented by

    \begin{equation*}
        \sheep_{\fred \to \me} + \pounds 100_{\me \to \fred}.
    \end{equation*}
\end{example}
