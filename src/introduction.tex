\section{Introduction}

To run a successful business, you need to keep track of your transactions\footnotemark,
\footnotetext{Of course, as I write this, I wonder if it's possible to set up a business with certain local rules, as to make a global set of accounts unnecessary, but I digress.}
so as to know how much capital, and goods, you have available to you.
You do this, as you cannot sell any goods, if you don't know that you have any to sell in the first place.

Even if you didn't need to do this, you'd still need to keep track of your transactions,
to know that your business is, in fact, successful.

Double entry bookkeeping, as formalized by Luca Pacioli, in 1494, \cite{pacioli}, is one method of doing this.
Many businesses have used this method of bookkeeping, largely unchanged, from ancient times \cite{encyclopedia-britannica}, 'till the modern day.

Double entry bookkeeping provides you with a collection of disconnected mathematical objects, and rules to manipulate these objects.
The aim of this paper is to build a mathematical bookkeeping system from the ground up,
to include these objects, and encode these rules, so as to unify the system,
and allow mathematical analysis of the whole.
