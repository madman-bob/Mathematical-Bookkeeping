\section{Conclusion}

This provides the beginnings of a formalized treatment of mathematical bookkeeping,
which allow us to describe a bookkeeping system, and prove common accounting results.

We should note that this formalization is far from complete, though.
For example, we have made no formal definition of debits and credits.
Indeed, everything we've done in this paper can be phrased without mentioning either of those two words.
We could introduce this concept by adding a lattice ordering to value spaces, and then prove results about debits and credits.

Similarly, we have made no distinction between different sorts of accounts,
so we cannot even state, let alone prove, the accounting equation.
Though once we have, it follows easily from the total balance of the balance account being $0$.

We have barely touched on the formal definition of time, defining it only as a total order,
and glossing over the operations required for continuous time appreciations.
What continuity restrictions do we want on transactions, and do we want any completeness restrictions on time?

If we go so far as to encode our stock control in our bookkeeping system,
then we can no longer trivially do a business valuation.
We could instead introduce valuation functions, to measure the value of our business, as linear maps between value spaces.

Finally, for real world applications, we need to consider how these objects would be represented in computer systems,
and what algorithms we would use to manipulate them.
