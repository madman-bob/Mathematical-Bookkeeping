\section{More Transaction Types}

Temporal transactions can represent in a single object things that would be represented by multiple transactions classically.
For example, consider straight-line appreciation, where we appreciate with transaction $t$ at equally spaced times $\tau_1$, $\tau_2$, ..., $\tau_n$.
This could be represented by a collection of step transactions $\left\{t^{\tau_i}\right\}_{i = 1}^n$,
or it could also be represented by a single temporal transaction $t'$, given by
\begin{equation*}
    t'(\tau) = \sum_{\tau_i} t^{\tau_i}(\tau).
\end{equation*}

If we increase the frequency of the step transactions, while keeping the final change the same,
we can approximate the appreciation occurring continuously with time.

Indeed, given appropriate operations on times,
we can define a continuous straight-line appreciation $l$ of value $a$, from time $\tau_1$ to $\tau_2$,
as a function $\Ti \to \V$, such that $l$ at time $\tau$ is given by

\begin{equation*}
    l(\tau) =
    \begin{cases}
        0 & \tau < \tau_1 \\
        \frac{\tau - \tau_1}{\tau_2 - \tau_1} a & \tau_1 \le \tau < \tau_2 \\
        a & \tau \ge \tau_2
    \end{cases}.
\end{equation*}

Other forms of appreciation, depreciation, and amortization may be represented classically as a collection of step transactions,
and so may also be represented as a single temporal transaction by adding together the component step transactions.

By taking smaller and smaller time steps, we may then approximate the continuous time equivalents.
These continuous time equivalents may usually be represented by simple functions $\Ti \to \V$.
For example, straight-line is a linear function on it's period of influence,
double-declining is piecewise exponential and linear,
sum-of-years-digits is quadratic,
and so on.

Notably this method doesn't work for units-of-production depreciation, as the amount to depreciate by is not known in advance.
