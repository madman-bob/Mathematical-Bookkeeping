\section{Account Personification}

The use of a sheep as a money type doesn't seem to make sense from the classical accounting point of view,
as a sheep is not a persistent store of value.
If I have a sheep, and the sheep dies, then there is no longer a sheep.
So how can we represent it in our bookkeeping system, if one day it will disappear?

This seeming contradiction can be resolved by considering the death of the sheep as another sort of transaction.
If we personify of the process of death, say by creating an account called $\death$,
then we can represent the death of the sheep by transferring the sheep from ourselves, to $\death$.

\begin{example}
    The death of a sheep could be represented by
    \begin{equation*}
        \sheep_{\me \to \death}.
    \end{equation*}
\end{example}

We can also use this idea for other sorts of processes.

\begin{example}
    As an example where we are receiving something from a process, consider lambing season.
    \begin{equation*}
        10 \sheep_{\operatorname{Mother\,Nature} \to \me}
    \end{equation*}
\end{example}

We can even have a transaction where we exchange some things for others.

\begin{example}
    For an example transaction with a process with movement in both directions,
    consider the construction of a chair.
    \begin{equation*}
        (
        3 \operatorname{Planks\,of\,wood} +
        4 \operatorname{Nails} +
        5 \operatorname{Man\,Hours}
        )_{\me \to \operatorname{Carpentry}} +
        1 \operatorname{Chair}_{\operatorname{Carpentry} \to \me}
    \end{equation*}
\end{example}

This allows you to encode your stock control into your bookkeeping system, by setting up a money type for each type of stock.
By increasing the level of detail you use for your money types, you can record more information about your stock.
For example, for the sheep above, you could encode them as Animals, as Sheep,
or you could even set up a money type for every individual sheep, allowing you to track each one precisely.
