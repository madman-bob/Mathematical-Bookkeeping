\documentclass[11pt]{article}

\usepackage{import}
\usepackage{a4wide}
\usepackage[english]{isodate}
\usepackage{amssymb}
\usepackage{amsmath}
\usepackage{amsthm}
\usepackage{url}
\usepackage{qtree}

\newcommand{\M}{\mathcal{M}} % Money space
\newcommand{\V}{\mathcal{V}} % Value space
\newcommand{\A}{\mathcal{A}} % Accounts
\newcommand{\Tr}{\mathcal{T}} % Transactions
\newcommand{\Ti}{\mathbb{T}} % Time
\renewcommand{\L}{\mathcal{L}} % Ledger

\newcommand{\taccount}[2]{\left[\,#1\,/\!/\,#2\,\right]}

\DeclareMathOperator{\sheep}{Sheep}

\DeclareMathOperator{\me}{Me}
\DeclareMathOperator{\nme}{Not~Me}
\DeclareMathOperator{\fred}{Fred}
\DeclareMathOperator{\death}{\textsc{Death}}

\newtheorem{definition}{Definition}
\newtheorem{notation}{Notation}
\newtheorem{theorem}{Theorem}
\newtheorem{example}{Example}

\setlength{\parindent}{0em}
\setlength{\parskip}{0.5em}

\title{A Mathematical Model of Financial Bookkeeping}
\author{Robert Wright}
\date{18th May 2018}

\makeatletter
\usepackage[pdftex,
pdfauthor={\@author},
pdfusetitle]{hyperref}

\begin{document}
    \abovedisplayskip=12pt
    \belowdisplayskip=12pt
    \abovedisplayshortskip=0pt
    \belowdisplayshortskip=7pt

    \begin{titlepage}
        \maketitle

        \begin{abstract}
            In this paper we introduce a mathematical model for an entire bookkeeping system.
            Building on the work of Ellerman, we define objects representing value, accounts, and transactions,
            and introduce a convenient notation for each of the above.
            This notation allows you to define transactions locally \textendash{} without knowing, a priori,
            all possible value types, and accounts.
            We then use these mathematical objects to formally prove known accounting results.

            We also extend transactions to temporal transactions, add a tree structure to the set of accounts,
            and create an object representing an entire ledger.
            These allow you to express as single objects financial events that are conceptually a single event,
            but must be expressed as multiple objects in standard financial bookkeeping
            (such as depreciation).
        \end{abstract}
    \end{titlepage}

    \import{./}{introduction.tex}
    \import{./}{value-space.tex}
    \import{./}{accounts.tex}
    \import{./}{transactions.tex}
    \import{./}{double-entry.tex}
    \import{./}{time.tex}
    \import{./}{ledgers.tex}
    \import{./}{value-space-2.tex}
    \import{./}{account-personification.tex}
    \import{./}{transactions-2.tex}
    \import{./}{accounts-tree.tex}
    \import{./}{conclusion.tex}

    \bibliographystyle{plain}
    \bibliography{bibliography}
\end{document}
